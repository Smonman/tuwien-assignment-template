\documentclass[10pt]{article}
\usepackage{tuwienassignment}

\usepackage{listings}
\usepackage{lipsum} % dummy text
\lstset{
    basicstyle=\ttfamily
}
\newcommand{\environmentcmd}[1]{\par\noindent\textbf{Environment:} \lstinline^#1^}
\newcommand{\commandcmd}[1]{\par\noindent\textbf{Command:} \lstinline^\\#1^}
\newcommand{\commandscmd}[2]{\par\noindent\textbf{Commands:} \lstinline^\\#1^ and \lstinline^\\#2^}

%%%
%%% PLACE CUSTOM STYLES HERE
%%%

\selectlanguage{english}

\title{Title}
\author{Author}
\matrikelnr{12345678}

\begin{document}
\maketitle

\section{Introduction}
\lipsum[1-1]

\subsection{Subsection}
\lipsum[2-3]

\subsubsection{Subsubsection}
\lipsum[2-3]

\paragraph{Paragraph}
\lipsum[1-2]

\subparagraph{Subparagraph}
\lipsum[1-2]

\section{Itemize and Enumerate}
\subsection{Itemize}
\begin{itemize}
  \item I am an item
  \item I am an item
  \item I am an item
        \begin{itemize}
          \item I am an item
          \item I am an item
          \item I am an item
                \begin{itemize}
                  \item I am an item
                  \item I am an item
                  \item I am an item
                \end{itemize}
        \end{itemize}
\end{itemize}

\subsection{Enumerate}
\begin{enumerate}
  \item I am an item
  \item I am an item
  \item I am an item
        \begin{enumerate}
          \item I am an item
          \item I am an item
          \item I am an item
                \begin{enumerate}
                  \item I am an item
                  \item I am an item
                  \item I am an item
                \end{enumerate}
        \end{enumerate}
\end{enumerate}

\newpage
\section{Exercises and Solutions}
\environmentcmd{exercise} produces a numbered exercise, which prefixes the current section on their counter. The section can correspond to each exercise sheet for example.

\environmentcmd{solution} produces a numbered solution. A solution has to be placed \emph{inside} of the exercise. Solutions will be numbered accordingly inside of an exercise environment. The number for a solution can be changed by writing the macro \lstinline^\setcounter{solutioncounter}{x}^ where \lstinline^x^ is the number the counter should be set to.

The numberformat can also be changed by calling the macro \lstinline^\renewcommand{\thesolutioncounter}{x}^ where \lstinline^x^ is the format. For example for roman numerals it would be \lstinline^\roman{solutioncounter}^.

\begin{exercise}
  \lipsum[1-1]
  \begin{solution}
    \lipsum[1-1]
  \end{solution}
  \begin{solution}
    \lipsum[1-1]
  \end{solution}
  \begin{solution}
    \lipsum[1-1]
  \end{solution}
\end{exercise}

\begin{exercise}[title=Übung]
  \lipsum[1-1]
  \begin{solution}
    \lipsum[1-1]
  \end{solution}
\end{exercise}

\begin{exercise}
  \renewcommand{\thesolutioncounter}{(\roman{solutioncounter})}%
  \lipsum[1-1]
  \begin{solution}
  \end{solution}
  \begin{solution}
  \end{solution}
\end{exercise}

\section{Math}
\subsection{Abs and Norm}
\commandcmd{abs} produces the following:
\begin{itemize}
  \item \emph{non-starred}
        \[\abs{a}\]
        \[\abs{\frac{a}{2}}\]
  \item \emph{starred}
        \[\abs*{a}\]
        \[\abs*{\frac{a}{2}}\]
\end{itemize}

\commandcmd{norm} produces the following:
\begin{itemize}
  \item \emph{non-starred}
        \[\norm{a}\]
        \[\norm{\frac{a}{2}}\]
  \item \emph{starred}
        \[\norm*{a}\]
        \[\norm*{\frac{a}{2}}\]
\end{itemize}

\subsection{Ceil and Floor}
\commandcmd{ceil} produces the following:
\begin{itemize}
  \item \emph{non-starred}
        \[\ceil{a}\]
        \[\ceil{\frac{a}{2}}\]
  \item \emph{starred}
        \[\ceil*{a}\]
        \[\ceil*{\frac{a}{2}}\]
\end{itemize}

\commandcmd{floor} produces the following:
\begin{itemize}
  \item \emph{non-starred}
        \[\floor{a}\]
        \[\floor{\frac{a}{2}}\]
  \item \emph{starred}
        \[\floor*{a}\]
        \[\floor*{\frac{a}{2}}\]
\end{itemize}

\subsection{Sets and Tupels}
\commandcmd{mset} produces the following:
\begin{itemize}
  \item \emph{non-starred}:
        \[\mset{1, 2, 3, 4}\]
        \[\mset{a \mid \frac{a}{2} > 5}\]
        \[\mset{}\]
        \[\mset{\mset{a}}\]
  \item \emph{starred}:
        \[\mset*{1, 2, 3, 4}\]
        \[\mset*{a \mid \frac{a}{2} > 5}\]
        \[\mset*{}\]
        \[\mset*{\mset*{a}}\]
\end{itemize}

\commandcmd{msetmid} produces the following:
\begin{itemize}
  \item \emph{non-starred}:
        \[\msetmid{x}{x > 0}\]
        \[\msetmid{\frac{1}{x}}{x > 0}\]
        \[\msetmid{}{}\]
        \[\msetmid{x}{x \in \mset{a}}\]
  \item \emph{starred}:
        \[\msetmid*{x}{x > 0}\]
        \[\msetmid*{\frac{1}{x}}{x > 0}\]
        \[\msetmid*{}{}\]
        \[\msetmid*{x}{x \in \mset{a}}\]
\end{itemize}

\commandcmd{mtupel} produces the following:
\begin{itemize}
  \item \emph{non-starred}:
        \[\mtupel{1, 2, 3, 4}\]
        \[\mtupel{\frac{1}{2}, \frac{2}{3}}\]
        \[\mtupel{}\]
  \item \emph{starred}:
        \[\mtupel*{1, 2, 3, 4}\]
        \[\mtupel*{\frac{1}{2}, \frac{2}{3}}\]
        \[\mtupel*{}\]
\end{itemize}

\subsection{Other}

\commandcmd{appliesto} produces the following with spacing:
\[\appliesto\]
which can be used e.g. in the following context:
\[\mathbf{n}(u) = \frac{\sum_{i = 0}^n w_i \mathbf{d}_i N_i^k(u)}{\sum_{i = 0}^n w_i N_i^k(u)} \appliesto u \in \left[u_0, u_{n + l}\right) \subset \mathbb{R}\]

\commandcmd{definedas} produces the following:
\[\definedas\]
which can be used e.g. in the following context:
\[\mathbf{b}_i^r \definedas (1 - t) \cdot \mathbf{b}_i^{r - 1} + t \cdot \mathbf{b}_{i + 1}^{r - 1}\]

\subsection{Conditions}
\environmentcmd{conditions}\footnote{see \url{https://tex.stackexchange.com/a/95842}} can be used for the following:
\\
Boltzmann distribution: state occupation probability of a thermodynamical system within fixed temperature \(T\): \[p(x) = \alpha \cdot e^{-\frac{E(x)}{k \cdot T}}\] where:
\begin{conditions}
  x      & \sep & state                                                                \\
  \alpha & \sep & degeneracy (= number of states \(x'\) with the same energy as \(x\)) \\
  E(x)   & \sep & energy                                                               \\
  k      & \sep & Bolzmann constant
\end{conditions}
It is possible to have different symbols instead of the dots:

\begin{conditions}[=]
  x      & \sep & state                                                                \\
  \alpha & \sep & degeneracy (= number of states \(x'\) with the same energy as \(x\)) \\
  E(x)   & \sep & energy                                                               \\
  k      & \sep & Bolzmann constant
\end{conditions}

\section{Logic}
\subsection{Verum and Falsum}
\commandcmd{ltrue} produces the following:
\[\ltrue\]
\commandcmd{lfalse} produces the following:
\[\lfalse\]

\section{Tables}
\subsection{Itemize in Tables}
\environmentcmd{tabitemize} creates an itemize environment inside of an minipage, so it can be placed in a table cell:

\begin{table}[!h]
  \centering
  \begin{tabularx}{0.5\linewidth}{Y}
    \toprule
    Header            \\
    \midrule
    \begin{tabitemize}
      \item item 1
      \item item 2
      \item item 3
    \end{tabitemize} \\
    \bottomrule
  \end{tabularx}
\end{table}

Please note, that this only works in \emph{tabularx} columns.

\section{Images}
Take a look at this example image: \footnote{Copyright free image from \url{https://pixabay.com/illustrations/background-pattern-lemon-texture-6703215/}}
\begin{figure}[h!t]
  \centering
  \includegraphics[scale=0.5]{happy_lemons.png}
  \caption{Happy Lemons}
  \label{fig:happylemons}
\end{figure}

\end{document}